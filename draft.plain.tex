1. Norm-Governed Decisions: A Computational Puzzle

People routinely make norm-consistent choices even when payoff, emotion,
and cognitive control all favor defection. We stop at red lights on
empty roads, keep unenforceable promises, and help strangers at our own
expense. These behaviors persist even when material incentives or
learned associations would push us the other way—suggesting an internal
mechanism that cannot be reduced to reward, affect, or generic control.

Most formal models of decision-making subsume norms under emergent
processes within general-purpose architectures:

- Utility models treat norms as high-value preferences
- Learning theories invoke conditioned associations via social feedback
- Dual-process accounts emphasize emotional inhibition or deliberative
  override
- Conflict-monitoring frameworks cast restraint as domain-general
  control

Under these views, norm-adherence arises “for free” as a byproduct of
utility maximization, social learning, or inhibitory control. Yet many
moral choices strike us as uncued, unsupervised, and unmotivated by
self-interest—prompting a different question:

  Could normative influence itself be a primitive component of the
  decision architecture?

To explore this, we introduce the Normative Executive System (NES), a
minimal extension of the Drift Diffusion Model (DDM) that encodes an
explicit norm weight (w_(n)) alongside standard salience and threshold
parameters. By embedding w_(n) directly into the drift equation, NES
operationalizes the hypothesis that norms exert a distinct, recoverable
pull on evidence accumulation—rather than emerging from other model
dimensions.

Recent reviews highlight a lack of empirical tests for whether explicit
normative representations are necessary to capture moral behavior
[@bello2023computationalapproachesto; @cushman2015moralconstraints]. NES
targets this gap by using Simulation-Based Calibration (ABC-SMC and
Neural Posterior Estimation) to ask three questions:

1.  Identifiability: Is w_(n) statistically recoverable from simulated
    behavior?
2.  Distinctiveness: Does NES generate decision patterns that standard
    DDMs cannot reproduce by parameter tuning alone?
3.  Structural Mismatch: Do fortified hierarchical DDMs systematically
    fail to recover w_(n) from NES-generated data?

By shifting the debate from philosophical intuition to statistical
identifiability and predictive utility, NES provides a formal, testable
framework for investigating whether normative self-governance is a
measurable, architecturally distinct faculty in human decision-making.

1.1 Related Work

NES builds on a rich tradition of cognitive modeling, including the
conflict monitoring framework [@Botvinick2001ConflictMonitoring],
dual-process theories of moral judgment [@Greene2001fMRI], and
attribute-wise value integration models [@Hare2009SelfControl]. However,
these models either lack explicit norm representations or do not
validate the identifiability of such constructs from behavior. NES
proposes a principled extension of the Drift Diffusion Model that treats
norm influence as an independent signal—recoverable and separable from
salience or utility.

2. The Normative Executive System: Formalizing Primitive Moral Architecture

2.1 Core Architectural Principle

The Normative Executive System (NES) extends the classic Drift Diffusion
Model by adding a dedicated norm weight (w_(n)) alongside the standard
salience weight (w_(s)). The drift rate is defined as

v = w_(s) (1 − λ) − w_(n) λ

where λ ∈ [0, 1] indexes conflict between stimulus salience and
internalized norms. Setting w_(n) apart as its own parameter creates an
oppositional architecture in which norms exert a direct, quantifiable
pull on evidence accumulation rather than being folded into existing
utility or control signals.

2.2 Why This Tests Computational Primitiveness

This simple formulation makes three critical tests possible:

1.  Oppositional Architecture: By pitting w_(s) vs. w_(n) in direct
    competition, we force the model to resolve norm–salience conflict
    explicitly.
2.  Parameter Independence: Because w_(n) is not a transform of other
    variables, its recoverability signals a true architectural
    primitive.
3.  Graded Conflict: Continuous variation in λ lets us probe whether the
    model responds systematically across low to high conflict—a hallmark
    of dedicated processing.

2.3 Architectural Predictions

NES yields three testable predictions that distinguish it from
emergentist accounts:

1.  Behavioral Signatures: Unique reaction-time and error-rate patterns
    (e.g., non-monotonic RT under high λ, systematic response
    suppression) that standard DDMs cannot mimic.
2.  Statistical Identifiability: The norm weight w_(n) should be
    recoverable from choice/RT data with minimal ambiguity.
3.  Architectural Irreducibility: Models lacking an explicit w_(n)
    should systematically fail to capture key norm-driven dynamics when
    fit to NES-generated data.

2.4 Simulation & Inference Overview

To evaluate these predictions, we implemented NES in a standard DDM
framework (see Supplement A for full details). Briefly:

- Task: A parametric “Stroop-like” conflict paradigm with five λ levels
  and moderate trial counts to test identifiability under realistic
  session lengths.
- Simulation: Euler–Maruyama integration generates choice/RT data under
  known (w_(s), w_(n), λ) settings.
- Inference: We apply both ABC-SMC and Neural Posterior Estimation (NPE)
  to assess recovery of w_(n) (Supplement B).
- Benchmark: A fortified hierarchical HDDM model—with per-condition
  drift regressors—tests whether standard approaches can indirectly
  recover w_(n).

2.5 Key Validation Steps

1.  Identifiability: SBC rank‐histograms for both ABC-SMC and NPE should
    approximate uniformity, indicating well-calibrated recovery of
    w_(n).
2.  HDDM Failure: If HDDM’s drift regressions produce biased or
    uncalibrated w_(n) estimates, this confirms an architectural
    mismatch.
3.  Behavioral Signatures: Simulated RT/error curves under varying w_(n)
    should reveal patterns (e.g., conflict-conditioned suppression)
    unique to NES.

By keeping the core drift equation and predictions front-and-center—and
delegating algorithmic details, parameter settings, and
summary‐statistic choices to the Supplement—this section emphasizes
NES’s conceptual contribution while reserving technical depth for
specialist readers.

2.6 HDDM Recovery of w_(n)

2.5.1 Pipeline Validation (Standard DDM Data)

Before analyzing NES data, we first validated our HDDM implementation on
standard DDM-simulated data. Five synthetic subjects each completed 3000
trials generated with true values of a = 1.5, v = 0.5, and t₀ = 0.2. The
recovered posterior distributions are shown in Figure 1.

All parameters were accurately recovered, with posterior mass
concentrated around ground truth values and all estimates falling within
5% of their respective true values. This confirms that the HDDM stack is
functioning as expected under its own assumptions.

[Posterior distributions for a, v, and t₀ recovered by HDDM on standard
DDM-simulated data.]

2.5.2 Fortified HDDM on NES Data

We next tested whether HDDM could indirectly recover w_(n) when applied
to NES-simulated data. To give HDDM every possible advantage, we
implemented a fortified hierarchical model with per-condition drift
regressors. The setup was as follows:

- True values: w_(n) ∼ Uniform(0.1, 2.0)
- Simulated data: NES-DDM with 5 subjects × 1000 trials; 5 λ levels
- Fit model: HDDM with group-level estimates of a, t₀, and v(λ)
- Derived w_(n): slope from regression of drift rates on conflict
- SBC: ranks of inferred w_(n) vs. ground truth

The results were unambiguous: - Parameter recovery was poor: Pearson
r = 0.29–0.62, R² = 0.05–0.38
- Systematic bias was observed (e.g., v_(0.5) bias = −0.48)
- Coverage within ±0.1 was only 21%–35%
- SBC ranks were maxed out (rank = 1000), indicating severe
overconfidence and miscalibration

[Comparison of true vs. estimated drift rates across conflict levels
from HDDM applied to NES-simulated data. Each point represents one SBC
iteration (subject-averaged group estimate). Systematic deviation from
the identity line indicates consistent misrecovery of drift rate
dynamics due to architectural mismatch.]

[Histogram of estimation error (estimated – true) for each drift
coefficient. Biases are visible in most conflict levels, especially at λ
= 0.5 and λ = 1.0. This structured error pattern confirms that HDDM
cannot represent the principled normative effects encoded by NES.]

  -------------------------------------------------------------------------
  Parameter                  r      R²     Bias    Std Err   ±0.1 Coverage
  -------------------------- ------ ------ ------- --------- --------------
  v_(Intercept)              0.62   0.38   -0.15   0.28      0.35

  v_([0.25])                 0.57   0.32   -0.22   0.30      0.30

  v_([0.5])                  0.42   0.18   -0.48   0.44      0.25

  v_([0.75])                 0.38   0.15   -0.31   0.33      0.23

  v_([1.0])                  0.29   0.05   -0.27   0.37      0.21
  -------------------------------------------------------------------------

Table 1: Drift rate regression recovery metrics from HDDM applied to
NES-simulated data (20 SBC iterations).

HDDM fails not due to noise, but because it lacks an architectural slot
for norm weighting—treating drift rate variations as unstructured. This
mismatch explains its systematic failure to recover w_(n) from
NES-simulated data.

2.7 Neural Posterior Estimation (NPE) for Multi-Parameter SBC

To assess the joint identifiability of core Minimal NES parameters and
to leverage potentially more efficient inference, we conducted
Simulation-Based Calibration (SBC) using Neural Posterior Estimation
(NPE). This approach aimed to recover effective norm weight (w_(n_eff)),
threshold (a), non-decision time (t), and effective salience (w_(s)).

The inference stack and SBC procedure were as follows:

- Parameters of Interest & Priors: For each SBC iteration i, a true
  parameter vector
  θ_(true)^((i)) = (w_(n_eff, true)^((i)), a_(true)^((i)), t_(true)^((i)), w_(s, true)^((i)))
  was drawn from a joint prior. Based on pilot recovery studies
  (run_parameter_recovery_minimal_nes_npe.py), these were:
  - w_(n_eff) ∼ Uniform(0.1, 2.0)
  - a ∼ Uniform(0.4, 1.5)
  - t ∼ Uniform(0.05, 0.5)
  - w_(s) ∼ Uniform(0.2, 1.5)
- Simulator: The NES-derived DDM (as per Section 2.1, with σ = 1.0 and
  T_(max) = 10.0s from BASE_SIM_PARAMS_RECOVERY used in
  run_parameter_recovery_minimal_nes_npe.py) generating choice/RT for
  the 5-level Stroop-like task (as per Section 2.2, N = 300 trials).
- Fixed DDM Simulation Constants: Noise σ = 1.0, dt = 0.01s,
  T_(max) = 10.0s. The parameters w_(n)_eff, a, t, w_(s) were the free
  parameters drawn from priors.
- Summary Statistics: The same comprehensive set of conditional and
  normalized summary statistics as described in Section 2.3 were used as
  input to the NPE.
- Inference Method: Neural Posterior Estimation (NPE), specifically
  SNPE-C (which uses Masked Autoregressive Flows as the density
  estimator), implemented via the sbi Python package
  [@sbi_package_tejero_etal_2020; @sbi_package_cranmer_etal_2020].
- NPE Training: A single NPE model was trained once on
  N_(train_sims) = 10, 000 simulations. Each simulation involved drawing
  a parameter vector θ from the joint prior and generating a
  corresponding vector of summary statistics x. The NPE was trained on
  these (θ, x) pairs. Training converged successfully after
  approximately [Avg_Epochs_NPE_Final_Run] epochs.
- Posterior Sampling: For each of the N_(SBC) = 100 “observed” datasets
  in the SBC loop, N_(post_samples) = 1000 samples were drawn from the
  trained NPE posterior p(θ|x_(obs)).
- Rank Calculation: For each SBC iteration i and for each of the four
  parameters k ∈ {w_(n_eff), a, t, w_(s)}, the rank was computed as:
  $$
  \mathrm{rank}^{(i)}_k = \sum_{j=1}^{1000} \mathbf{1}\left(\theta_{k,\mathrm{post}}^{(i,j)} \leq \theta_{k,\mathrm{true}}^{(i)}\right)
  $$

[Simulation-Based Calibration (SBC) ECDFs for NES parameters using
Neural Posterior Estimation (NPE). Each panel shows the empirical
cumulative distribution function (ECDF) of posterior ranks (blue),
bounded by the 95% beta confidence interval for uniformity (red dashed).
The diagonal line represents ideal calibration. All parameters show
approximately uniform rank distributions, with w_(n), w_(s), and t₀
exhibiting especially strong calibration. These results confirm the
joint identifiability and inferential precision of the NES model under
NPE.]

2.8 Implementation Details

All simulations and inference procedures were run on a workstation with
40GB RAM and an NVIDIA RTX GPU. NES simulations used Python 3.10, NumPy,
and custom Euler–Maruyama integration. ABC-SMC was implemented via pyABC
[@Klinger2018pyABC]. Neural Posterior Estimation (NPE) used the sbi
library [@sbi_package_tejero_etal_2020], with SNPE-C and masked
autoregressive flows trained for ~120 epochs using Adam (lr=1e-4). Code
for simulations, summary statistic extraction, and inference will be
made available upon request or upon publication.

3. Testing Computational Primitiveness Through Parameter Recovery

3.1 Methodological Strategy

Our approach to testing computational primitiveness proceeds through
three phases, each designed to provide converging evidence for the
primitive nature of normative influence:

1.  Phase 1: Demonstrate w_(n) identifiability using simulator-based
    inference (ABC-SMC, NPE)
    - Tests whether normative influence can be reliably recovered from
      behavioral data
    - Uses multiple inference methods to ensure robustness
    - Assesses parameter recovery under realistic conditions
2.  Phase 2: Show systematic failure of hierarchical DDM approaches
    - Demonstrates that standard models cannot capture norm-driven
      behavior
    - Highlights the architectural mismatch between emergentist and
      primitive accounts
    - Provides negative evidence against purely emergentist explanations
3.  Phase 3: Characterize unique behavioral signatures of normative
    influence
    - Identifies patterns specific to norm-driven behavior
    - Shows these patterns cannot be mimicked by utility or control
      parameters
    - Provides positive evidence for the primitive nature of normative
      influence

This progression directly tests the core prediction that if norms are
computationally primitive, they should be identifiable through
appropriate methods but invisible to methods that assume an emergent
architecture. The following sections present the results of each phase,
with methodological details provided in the corresponding sections.

3.2 The Logic of Simulation-Based Calibration

To test whether normative influence is computationally primitive, we
must demonstrate that:

1.  NES parameters are identifiable: If w_(n) reflects real
    computational processes, it should be recoverable from behavioral
    data
2.  Standard models fail systematically: If norms are primitive (not
    emergent), existing frameworks should show architectural mismatch
    when applied to norm-driven data
3.  Behavioral signatures are unique: Normative influence should produce
    patterns that utility/control models cannot mimic

Simulation-Based Calibration (SBC) provides the ideal framework for
testing these claims under controlled conditions.

3.3 Phase 1: Identifiability via Simulator-Based Inference

ABC-SMC Recovery of w_(n)

We first assessed the identifiability of w_(n) using Approximate
Bayesian Computation with Sequential Monte Carlo (ABC-SMC). Across 100
simulated datasets with 150 posterior samples each, we observed
excellent calibration of the posterior estimates, with the rank
histogram closely approximating uniformity (χ²(14) = 22.95, p = 0.061).
This indicates that the ABC-SMC pipeline yields well-calibrated
posterior estimates without systematic bias.

[SBC rank histogram for w_(n), showing approximately uniform rank
distribution across 100 iterations.]

Neural Posterior Estimation (NPE) Results

To complement the ABC-SMC approach and address its limitations, we
implemented Neural Posterior Estimation (NPE) for multi-parameter
recovery. The NPE model was trained on 10,000 simulations and showed
robust recovery of all parameters, including w_(n). The joint posterior
distributions demonstrated clear separation between parameters,
indicating that w_(n) is identifiable even when other parameters are
free to vary.

Robustness Under Parameter Jitter

To assess identifiability under realistic uncertainty, we introduced
±10% uniform jitter to fixed parameters (a, t₀, w_(s)) across 50 SBC
runs. Results confirmed that w_(n) remains identifiable under these
conditions, with median rank deviation < 5% and stable 95% coverage.
This robustness supports NES’s applicability in settings with individual
variation.

3.4 Phase 2: Architectural Failure of HDDM

The failure of standard hierarchical DDMs (HDDMs) to recover w_(n)
provides strong evidence for the architectural distinctness of normative
influence. When we applied HDDM with regression over conflict levels to
NES-generated data, we observed systematic failures in parameter
recovery:

- Poor parameter recovery: Pearson r = 0.29–0.62, R² = 0.05–0.38
- Systematic bias: Consistent underestimation of drift rates across
  conflict levels
- Miscalibrated uncertainty: SBC ranks were consistently at maximum,
  indicating systematic overconfidence

[HDDM rank histogram for w_(n) estimates. Ranks clustered at the maximum
value indicate systematic miscalibration, reflecting HDDM’s inability to
capture the normative component.]

This failure is not due to insufficient data or model flexibility, but
rather reflects a fundamental architectural mismatch—HDDM lacks the
representational capacity to capture the normative gating mechanism
implemented in NES.

3.5 Phase 3: Unique Behavioral Signatures of w_(n)

Distinct Response Patterns

High values of w_n produce a characteristic pattern of decreasing RTs
and error rates with increasing conflict—a signature that cannot be
explained by standard decision variables. Figure 4 shows how different
values of w_n lead to distinct behavioral profiles across conflict
levels.

[Behavioral signatures of the norm weight w_(n) across conflict levels.
Left: Error rate decreases more steeply with conflict for higher w_(n),
indicating stronger suppression of salience-driven responses. Right:
Mean correct response times (RTs) also decrease with higher w_(n),
suggesting faster commitment in norm-congruent decisions. These patterns
are not replicable by standard DDMs and reflect the architectural
distinctiveness of NES.]

Conflict-Conditioned Error Rates

A key prediction of the NES framework is that high w_n values should
lead to decreasing error rates at higher conflict levels—a pattern not
predicted by standard decision models. This prediction was confirmed in
our simulations (Figure 5).

[Left: Error rates by conflict level (λ) across five parameter regimes.
Only the high_wn condition exhibits a strong monotonic suppression of
errors with increased conflict. Right: Joint behavioral slope profiles
(RT vs. error rate). The high_wn point lies in the lower-left quadrant,
combining decreasing RTs and decreasing errors with conflict—a signature
that no other parameter combination replicates. These results
demonstrate the behavioral distinctiveness of w_(n) and reject
equifinality from alternative DDM parameterizations.]

Equifinality Analysis

We tested whether combinations of other parameters could mimic the
effects of w_n by simulating five distinct parameter regimes:

- Low w_n (0.2): impulsive, error-prone under high conflict
- Mid w_n (0.5–1.0): balanced adaptation
- High w_n (1.5–2.0): slower, norm-consistent responding

These regimes produced distinct and reproducible patterns in RT and
accuracy that could not be matched by threshold or salience weight
alone, confirming that w_n captures a unique dimension of behavioral
variation.

- Identifiability Proof: SBC rank‐histograms for both ABC-SMC and NPE
  approximate uniformity, confirming well-calibrated recovery of w_(n).
- Architectural Distinctness: HDDM’s systematic failure underscores that
  emergentist models cannot reproduce NES’s normative gating.
- Distinct Behavioral Patterns: Only NES yields the characteristic RT
  and accuracy signatures under graded conflict.

Together, these results converge to show that normative weight operates
as a mechanistically distinct and recoverable component of decision
dynamics—supporting the thesis that normative influence is a
computational primitive rather than an emergent artifact of
general‐purpose models.

4. Early Pilot Evidence for Human Framing Data

As an exploratory proof-of-concept, we applied a 5-parameter NES variant
(including an α_(gain) learning rate) to archival framing-choice data
from N = 45 participants (modeled after Roberts & Gershman, 2021). Using
the same NPE pipeline validated above, we extracted each subject’s mean
normative weight (w_(n)) and correlated it with their individual framing
susceptibility (risk preference difference between gain vs. loss
frames).

  Exploratory finding: Subjects with higher NES-inferred w_(n) tended to
  show a larger framing effect (r = 0.87, p < 0.0001; see Fig. 7).

Because this analysis was not pre-registered and lacks full experimental
details (participant exclusions, exact task timing, and comprehensive
model comparisons), we present it here as preliminary “teaser” evidence.
A dedicated empirical follow-up—complete with full methods, hierarchical
modeling, and formal model‐comparison metrics—is forthcoming
[@WrightInPrep].

5. Discussion

5.1 Key Takeaways

We provide the first simulation-based proof that a dedicated norm weight
(w_(n)) is both identifiable and functionally distinct in decision
architectures.
- Calibrated Recovery: SBC via ABC-SMC and NPE yields uniform rank
histograms for w_(n), confirming reliable inference under realistic
trial counts.
- Unique Behavioral Patterns: Varying w_(n) produces non-monotonic RT
curves and conflict-conditioned suppression of errors—signatures
standard DDMs cannot replicate.
- Architectural Gap: Fortified hierarchical HDDM systematically
misrecovers w_(n) (r=0.29–0.62, R²<0.4; extreme SBC skew), demonstrating
that emergentist models lack the structural slot for normative
influence.

5.2 Implications for Decision Modeling

- Emergent vs. Primitive: These results challenge views that
  norm-adherence “emerges” from value or control parameters alone.
  Instead, explicit normative gating appears necessary to capture moral
  dynamics.
- Necessity of Simulation-Based Inference: Traditional DDM fitting fails
  where ABC-SMC and NPE succeed, underscoring the importance of
  simulator-based methods when assessing models with structured
  components.

5.3 Mapping NES to Broader Theoretical Frameworks

The Normative Executive System (NES) can be situated within classical
cognitive science frameworks, particularly Marr’s three levels of
analysis. This alignment helps clarify what NES contributes above and
beyond existing value-based models, and offers testable bridges between
normative theory and mechanistic modeling.

Computational level (What is the goal?)

At this level, NES formalizes a distinct goal: to adhere to internalized
norms even when they conflict with salience or reward. The norm weight
parameter w_(n) quantifies the agent’s commitment to this goal. This is
conceptually akin to the inclusion of “deontic value” in some RL models,
but here it is treated as a separable decision influence rather than a
reweighted utility.

Algorithmic level (How is it computed?)

NES specifies a conflict-sensitive drift rate:

v = w_(s)(1 − λ) − w_(n)λ

This oppositional formulation makes normative influence explicit and
measurable, unlike standard DDMs where such dynamics are implicit. It
resembles cognitive control mechanisms (e.g., dACC-driven conflict
monitoring), but instantiates them within a principled decision rule
that can be empirically validated.

Implementational level (What systems realize this?)

NES offers testable neural predictions. Individuals with higher inferred
w_(n) may show stronger mid-frontal theta activity during norm conflict,
or distinct modulation of value-related regions (e.g., vmPFC) and
control systems (e.g., dlPFC) depending on w_(n) and conflict levels.
This bridges NES to neurocognitive models of moral cognition.

5.4 NES in Relation to Integrated Value-Based Accounts

A prominent alternative to NES is the integrated value-based framework,
where norms are treated as weighted attributes within a single
accumulation process. Attribute-wise drift diffusion models (anDDMs),
such as those developed by Hutcherson et al., model decisions as a
function of competing value dimensions (e.g., hedonic vs. normative),
without positing a dedicated normative faculty. These models replicate
neural and behavioral data, including dlPFC activity and norm-consistent
choices, through attentional dynamics and weight variation.

Similarly, Gautheron et al. demonstrate that moral behavior can emerge
from shared neural fields if normative evidence receives temporal
precedence or differential salience. These approaches show that a single
integrator, when tuned appropriately, can mimic normative behavior
without invoking architectural separation.

NES does not dispute the descriptive success of these models. Rather, it
highlights an inferential limitation: in tasks where norms conflict
directly with salience, unified DDMs—even when enhanced—systematically
fail to recover normative dynamics. This is evident in Section 2.5,
where HDDM was unable to recover w_(n) from NES-simulated data (Figures
2–3), despite using a flexible regression structure.

The key contribution of NES is not that it performs better on all tasks,
but that it isolates architectural necessity. If norms can be modeled
just by reweighting, then emergentist DDMs should be able to recover
w_(n). That they do not—despite extensive tuning—suggests that some
normative processes may demand explicit, dissociable representations.
NES provides a framework to test this claim with computational
precision.

5.5 Methodological Notes

- SBC Scope: Although SBC uses NES-generated data, it tests inferential
  calibration—not empirical validity. Full model-fit diagnostics,
  convergence curves, and distance-metric choices are detailed in
  Supplement A.
- Fixed Parameters: Piloted settings for a, w_(s), and t₀ aided
  identifiability; jitter analyses (±10%) confirm robustness, but
  hierarchical estimation with constrained priors could enhance
  ecological validity.

5.6 Practical Recommendations

1.  Design: Employ multiple, balanced conflict levels and ≥1000
    trials/participant.
2.  Inference: Use ABC-SMC or NPE rather than standard HDDM.
3.  Validation: Always run SBC on each pipeline before interpreting
    w_(n) estimates.

5.7 Future Directions

- Hierarchical NES: Jointly estimate group and individual w_(n) under
  hierarchical priors.
- Neural Validation: Link w_(n) to mid-frontal theta or fMRI markers of
  conflict.
- Applied Domains: Extend to clinical populations (ASD, OCD) and
  developmental studies of norm acquisition.

5.8 Conclusion

By demonstrating that w_(n) is recoverable, produces unique behavioral
signatures, and eludes standard DDMs, we establish normative influence
as a computational primitive. NES lays the groundwork for treating
concepts like duty and restraint as measurable forces in moral
cognition—a foundational advance in computational moral theory.

Full technical details and extended analyses are provided in the
Supplement.

Data and Code Availability

All code, preprocessed data, and containerized environments (Docker)
needed to reproduce every figure and analysis will be made publicly
available via [GitHub URL] and archived on Zenodo under DOI [Zenodo
DOI].

Appendix A: Summary Statistics

A.1 Complete List of Summary Statistics

For each of the five conflict levels (λ), we computed the following:

- Error rate
- Correct RTs:
  - Mean
  - Median
  - Variance
  - 25th percentile
  - 75th percentile
  - Skewness
  - Minimum
  - Maximum
  - Range
- Error RTs: Same statistics as above

All RT-based statistics were normalized by T_(max) = 4.0 seconds to
maintain consistent scale across metrics.

A.2 Distance Function Weights

We used a weighted L2 (Euclidean) distance metric across normalized
statistics, with weights chosen based on sensitivity to w_n:

- Error rate: 2.0 (highest priority)
- Correct RT mean & median: 1.5 (high priority)
- All other statistics: 1.0 (standard weight)

These weights were selected heuristically based on preliminary runs that
prioritized behavioral features most impacted by norm weighting. While
not systematically optimized, sensitivity analyses suggested robustness
within a 1:1–2:1 weight ratio.

A.3 Data Quality and Missing Values

NaN Handling:

- If both simulated and observed values were NaN: the statistic was
  excluded from the distance calculation.
- If one value was NaN and the other was not: a penalty of +100 was
  added to the distance to strongly discourage parameter regions
  producing undefined behavior (e.g., zero errors in a high-conflict
  condition).

This approach penalizes implausible parameter settings while preserving
numerical stability in the inference pipeline.

Appendix B: Glossary

- DDM (Drift Diffusion Model): A model of binary decision-making using
  evidence accumulation.

- Drift rate (v): The average rate at which evidence accumulates toward
  a decision.

- Threshold (a): The boundary distance that determines how much evidence
  is needed before making a decision.

- Non-decision time (t): The time consumed by processes other than
  decision-making (e.g., perception, motor response).

- Simulation-Based Calibration (SBC): A method to assess whether
  posterior inference is well-calibrated when ground truth is known.

- Neural Posterior Estimation (NPE): A machine learning technique for
  approximating posterior distributions without handcrafted distance
  metrics.

- Norm Weight (wₙ): A NES parameter representing the strength of
  normative influence on decisions.
